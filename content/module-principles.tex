\Chapter{Paketų architektūros principai}

Paketų architektūros principai nurodo kaip turėtų organizuojama klasių
struktūra didesnėse sistemose, kad jos būdų labiau organizuotos
ir lengviau valdomos. Principus galima suskirstyti į dvi grupes:
\begin{enumerate}
  \item \emph{Paketų rišlumo principai} padeda suprasti į kuriuos
    paketus, kurios klasės turėtų būti dedamos:
    \begin{enumerate}
      \item \nameref{subsection:package:rep}.
      \item \nameref{subsection:package:ccp}.
      \item \nameref{subsection:package:crp}.
    \end{enumerate}
  \item \emph{Paketų jungimo principai} padeda suprasti kaip paketa
    turėtų būti susieti vieni su kitais:
    \begin{enumerate}
      \item \nameref{subsection:package:adp}.
      \item \nameref{subsection:package:sdp}.
      \item \nameref{subsection:package:sap}.
    \end{enumerate}
\end{enumerate}

Plačiau: \url{http://en.wikipedia.org/wiki/Package\_Principles},
\cite{martin2000design}, \cite{martin1995designing},
\cite{martin2003agile}.

\section{Paketų rišlumo principai}

\subsection{Panaudojimo/išleidimo ekvivalentumo principas}

\label{subsection:package:rep}

\en{Reuse/Release Equivalence Principle, REP}.

Kodas, kuris yra kartu išleidžiamas yra kartu ir perpanaudojamas.

\subsection{Bendro uždarinio principas}

\label{subsection:package:ccp}

\en{Common Closure Principle, CCP}

Klasės, kurios yra linkusios keistis kartu, turėtų priklausyti
tam pačiam paketui.

\subsection{Bendro panaudojimo principas}

\label{subsection:package:crp}

\en{Common Reuse Principle, CRP}

Klasės, kurios nėra naudojamos kartu, neturėtų priklausyti tam
pačiam paketui.

Kai klientas priklauso nuo klasės, tai jis priklauso nuo paketo,
kurioje ta klasė yra. Jei išėjo nauja paketo versija, tai nepaisant
to, kad kliento naudojama klasė nepasikeitė, vis tiek klientas
turės būti patikrintas, todėl į paketus turėtų būti dedamos
tik tos klasės, kurios bus perpanaudojamos kartu.

\section{Paketų jungimo principai}

\subsection{Neciklinės priklausomybės principas}

\label{subsection:package:adp}

\en{Acyclic Dependency Principle, ADP}

Priklausomybės tarp paketų neturėtų būti ciklinės.

\subsection{Stabilios priklausomybės principas}

\label{subsection:package:sdp}

\en{Stable Dependency Principle, SDP}

Paketai turėtų priklausyti nuo stabilesnių už juos. Stabilumas, tai
kiek darbo reikia įdėti, kad padaryti pakeitimą.

Paketo nestabilumas:
\begin{equation*}
  I = \frac{Ce}{Ce + Ca},
\end{equation*}
čia:
\begin{description}
  \item[$Ca$] – įeinantys ryšiai (skaičius nepriklausančių paketui
    klasių, kurios priklauso nuo klasių esančių jo viduje);
  \item[$Ce$] – išeinantys ryšiai (skaičius nepriklausančių paketui
    klasių, nuo kurių priklauso jo viduje esančios).
\end{description}

Performulavus principą: paketas turi priklausyti nuo paketų, kurių
$I$ mažesnis nei jo.

\subsection{Stabilios abstrakcijos principas}

\label{subsection:package:sap}

\en{Stable Abstraction Principle, SAP}

Stabilus paketai turėtų būti abstraktūs.

Paketo abstraktumas:
\begin{equation*}
  A = \frac{Na}{Nc},
\end{equation*}
čia:
\begin{description}
  \item[$Nc$] – klasių kiekis pakete;
  \item[$Na$] – abstrakčių klasių kiekis pakete.
\end{description}

Paketo nuotolis nuo norimo varianto:
\begin{equation*}
  D' = |A + I - 1|.
\end{equation*}
