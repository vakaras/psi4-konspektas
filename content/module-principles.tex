\Chapter{Paketų architektūros principai}

Paketų architektūros principai nurodo kaip turėtų organizuojama klasių
struktūra didesnėse sistemose, kad jos būdų labiau organizuotos
ir lengviau valdomos. Principus galima suskirstyti į dvi grupes:
\begin{enumerate}
  \item \emph{Paketų rišlumo principai} padeda suprasti į kuriuos
    paketus, kurios klasės turėtų būti dedamos:
    \begin{enumerate}
      \item \nameref{section:package:rep}.
      \item \nameref{section:package:ccp}.
      \item \nameref{section:package:crp}.
    \end{enumerate}
  \item \emph{Paketų jungimo principai} padeda suprasti kaip paketa
    turėtų būti susieti vieni su kitais:
    \begin{enumerate}
      \item \nameref{section:package:adp}.
      \item \nameref{section:package:sdp}.
      \item \nameref{section:package:sap}.
    \end{enumerate}
\end{enumerate}

Plačiau: \url{http://en.wikipedia.org/wiki/Package\_Principles},
\cite{martin2000design}, \cite{martin1995designing},
\cite{martin2003agile}.

\section{Panaudojimo/išleidimo ekvivalentumo principas}

\label{section:package:rep}

\en{Reuse/Release Equivalence Principle, REP}.

\section{Bendro uždarinio principas}

\label{section:package:ccp}

\en{Common Closure Principle, CCP}

\section{Bendro panaudojimo principas}

\label{section:package:crp}

\en{Common Reuse Principle, CRP}

\section{Neciklinės priklausomybės principas}

\label{section:package:adp}

\en{Acyclic Dependency Principle, ADP}

\section{Stabilios priklausomybės principas}

\label{section:package:sdp}

\en{Stable Dependency Principle, SDP}

\section{Stabilios abstrakcijos principas}

\label{section:package:sap}

\en{Stable Abstraction Principle, SAP}
