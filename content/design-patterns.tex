\Chapter{Projektavimo schemos}

Plačiau: \cite{gamma1995design}, \cite{martin2003agile}
\url{http://en.wikipedia.org/wiki/Design\_pattern\_(computer\_science)}.

\section{Kūrimo schemos}

\en{Creational patterns}

\subsection{Metodas gamintojas}

\en{Factory method}

Plačiau: \url{http://en.wikipedia.org/wiki/Factory\_method}.

\subsection{Abstrakti gamykla}

\en{Abstract factory}

Plačiau: \url{http://en.wikipedia.org/wiki/Abstract\_factory\_pattern}.

\subsection{Vienetinis objektas}

\en{Singleton}

Plačiau: \url{http://en.wikipedia.org/wiki/Singleton\_pattern}.

\section{Struktūrinės schemos}

\en{Structural patterns}

\subsection{Adapteris}

\en{Adapter}

Plačiau: \url{http://en.wikipedia.org/wiki/Adapter_pattern}.

\subsection{Tiltas}

\en{Bridge}

Plačiau: \url{http://en.wikipedia.org/wiki/Bridge\_pattern}.

\subsection{Dekoratorius}

\en{Decorator}

\section{Veiksenos schemos}

\en{Behavioral patterns}

\subsection{Būsena}

\en{State}

Plačiau: \url{http://en.wikipedia.org/wiki/State\_pattern}.

\subsection{Metodas šablonas}

\en{Template method}

\subsection{Strategija}

\en{Strategy}

\subsection{Lankytojas}

\en{Visitor}

Plačiau: \url{http://en.wikipedia.org/wiki/Visitor\_pattern}.
