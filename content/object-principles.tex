\Chapter{Objektiškai orientuoto projektavimo principai}

\human{Robertas C. Martinas} savo darbe \cite{martin2000design}
išskyrė penkis pagrindinius objektiškai orientuoto programavimo
principus, kurie sutrumpintai vadinami \strong{SOLID}:
\begin{enumerate}
  \item \emph{Vienos atsakomybės principas} \en{Single responsibility
    principle, SRP} – objektas turi būti atsakingas tik už vieną
    dalyką.
  \item \emph{Atvirumo-uždarumo principas} \en{Open/closed principle, OCP}
    – programinės įrangos esybės turi būti atviros praplėtimui, bet
    uždaros pakeitimui.
  \item \emph{\human{Liskovos} pakeitimo principas} \en{Liskov substitution
    principle, LSP} – objektai programoje turi būti pakeičiami
    jų potipių egzemplioriais, nedarant poveikio programos teisingumui.
  \item \emph{Sąsajų atskyrimo principas} \en{Intrerface segregation
    principle, ISP} – daug specialiuotų klientams sąsajų yra geriau
    nei viena bendro pobūdžio.
  \item \emph{Priklausomybių apgręžimo principas} \en{Dependency
    inversion priciple, DIP} – priklausyti nuo abstrakcijų, nepriklausyti
    nuo realizacijų.
\end{enumerate}

Plačiau: \url{http://en.wikipedia.org/wiki/SOLID},
\cite{martin2000design}, \cite{Grosberg1997design},
\cite{martin1995designing}, \cite{martin2003agile}.

\section{Vienos atsakomybės principas}

\url{http://en.wikipedia.org/wiki/Single_responsibility_principle}

\section{Atvirumo-uždarumo principas}

\url{http://en.wikipedia.org/wiki/Open/closed_principle}

\section{\human{Liskovos} pakeitimo principas}

\url{http://en.wikipedia.org/wiki/Liskov_substitution_principle}

\section{Sąsajų atskyrimo principas}

\url{http://en.wikipedia.org/wiki/Interface_segregation_principle}

\section{Priklausomybių apgręžimo principas}

\url{http://en.wikipedia.org/wiki/Dependency_inversion_principle}

\section{Generalizacijos-specializacijos santykis}

TODO

Generalizacijos-specializacijos santykis, kaip tėvinės klasės išplėtimas
ir apribojimas, specializacijos ribojant vengimas.

Atitinka PRE ir POST sąlygas tėvynėje ir vaikinėje klasėje.  
Pavyzdžiui, tėvynė (pre arg>5 rez >10), vaikinė (arg>3 rez>11) gerai.

\section{Konceptualių hierarchijų modeliavimas}

TODO: \cite{Grosberg1997design}.

Konceptualių hierarchijų modeliavimas, nepažeidžiant LSP
(išvengiant specializacijos ribojant).
